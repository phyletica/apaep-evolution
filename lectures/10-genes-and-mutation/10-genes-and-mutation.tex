% \documentclass[table]{beamer}
\documentclass[table,handout]{beamer}
\setbeameroption{show notes}
% \setbeameroption{hide notes}
% \setbeameroption{show only notes}
\usepackage{varwidth}

\newif\ifhide
\newif\ifpost
\newif\ifhideclicker

% \hidetrue
% \hideclickertrue
\posttrue

\newcommand{\whiteout}[1]{\textcolor{white}{#1}}
% \newcommand{\whiteoutbox}[1]{\fcolorbox{white}{white}{\parbox{\dimexpr \linewidth-2\fboxsep-2\fboxrule}{\whiteout{#1}}}}
% \newcommand{\notebox}[1]{\fcolorbox{blue}{white}{\parbox{\dimexpr \linewidth-2\fboxsep-2\fboxrule}{#1}}}
\newcommand{\whiteoutbox}[1]{\fcolorbox{white}{white}{\parbox{\linewidth}{\whiteout{#1}}}}
\newcommand{\notebox}[1]{\fcolorbox{blue}{white}{\parbox{\linewidth}{#1}}}
\newcommand{\blankbox}[1]{\phantom{\varwidth{\linewidth}\whiteoutbox{#1}\endvarwidth}}
\newcommand{\blank}[1]{\phantom{\varwidth{\linewidth}#1\endvarwidth}}

\ifhide%
    \newcommand{\hmask}[1]{\blank{#1}}%
\else%
    \newcommand{\hmask}[1]{#1}%
\fi

\ifhide%
    \newcommand{\cmask}[1]{\blank{#1}}%
\else%
    \newcommand{\cmask}[1]{\textcolor{blue}{#1}}%
\fi

\ifhide%
    \newcommand{\wout}[1]{\whiteout{#1}}%
\else%
    \newcommand{\wout}[1]{#1}%
\fi

\ifhide%
    \newcommand{\hignore}[1]{}%
\else%
    \newcommand{\hignore}[1]{#1}%
\fi

\ifpost%
    \newcommand{\nopost}[1]{}%
\else%
    \newcommand{\nopost}[1]{#1}%
\fi

\ifpost%
    \newcommand{\maskpost}[1]{\blank{#1}}%
\else%
    \newcommand{\maskpost}[1]{#1}%
\fi

\ifhideclicker%
    \newcommand{\clickerslide}[1]{\stepcounter{clickerQuestionCounter}%
        \begin{frame}[t]
            \textcolor{blue}{Q \arabic{clickerQuestionCounter}:}
        \end{frame}}
\else%
    \newcommand{\clickerslide}[1]{#1}%
\fi

\ifhideclicker%
    \newcommand{\clickerpost}[1]{\stepcounter{clickerQuestionCounter}%
        \begin{frame}[t]
            \textcolor{blue}{Q \arabic{clickerQuestionCounter}:}
        \end{frame}}
\else%
    \newcommand{\clickerpost}[1]{\stepcounter{clickerQuestionCounter}#1}%
\fi

\ifhide%
    \newcommand{\hidebox}[1]{\blank{#1}}%
\else%
    \newcommand{\hidebox}[1]{\notebox{#1}}%
\fi

\ifhide%
    \newcommand{\wbox}[1]{\whiteoutbox{#1}}%
\else%
    \newcommand{\wbox}[1]{\notebox{#1}}%
\fi

\ifhide%
    \newcommand{\nbox}[1]{\blankbox{#1}}%
\else%
    \newcommand{\nbox}[1]{\notebox{#1}}%
\fi

\ifhideclicker%
    \newcommand{\clickeranswer}[1]{#1}%
\else%
    \ifhide%
        \newcommand{\clickeranswer}[1]{#1}%
    \else%
        \newcommand{\clickeranswer}[1]{\textbf{\textcolor{blue}{#1}}}%
    \fi
\fi

\usepackage{beamerthemesplit}
% \usetheme{boxes}
\usetheme{Malmoe}
\usecolortheme{seahorse}
% \usecolortheme{seagull}
\usepackage{ifthen}
\usepackage{xspace}
\usepackage{multirow}
\usepackage{multicol}
\usepackage{booktabs}
\usepackage{xcolor}
\usepackage{wasysym}
\usepackage{comment}
\usepackage{hyperref}
\hypersetup{pdfborder={0 0 0}, colorlinks=true, urlcolor=blue, linkcolor=blue, citecolor=blue}
\usepackage{changepage}
\usepackage[compatibility=false]{caption}
\captionsetup[figure]{font=scriptsize, labelformat=empty, textformat=simple, justification=centering, skip=2pt}
\usepackage{tikz}
\usepackage{pgf}
\usetikzlibrary{trees,calc,backgrounds,arrows,positioning,automata}

\usepackage{array}
\newcolumntype{L}[1]{>{\raggedright\let\newline\\\arraybackslash\hspace{0pt}}p{#1}}
\newcolumntype{C}[1]{>{\centering\let\newline\\\arraybackslash\hspace{0pt}}p{#1}}
\newcolumntype{R}[1]{>{\raggedleft\let\newline\\\arraybackslash\hspace{0pt}}p{#1}}

\usepackage[bibstyle=joaks-slides,maxcitenames=1,mincitenames=1,backend=biber]{biblatex}

\newrobustcmd*{\shortfullcite}{\AtNextCite{\renewbibmacro{title}{}\renewbibmacro{in:}{}\renewbibmacro{number}{}}\fullcite}

\newrobustcmd*{\footlessfullcite}{\AtNextCite{\renewbibmacro{title}{}\renewbibmacro{in:}{}}\footfullcite}

% Make all footnotes smaller
% \renewcommand{\footnotesize}{\scriptsize}

\definecolor{myGray}{gray}{0.9}
\colorlet{rowred}{red!30!white}

\setbeamertemplate{blocks}[rounded][shadow=true]

\setbeamercolor{defaultcolor}{bg=structure!30!normal text.bg,fg=black}
\setbeamercolor{block body}{bg=structure!30!normal text.bg,fg=black}
\setbeamercolor{block title}{bg=structure!50!normal text.bg,fg=black}

\newenvironment<>{varblock}[2][\textwidth]{%
  \setlength{\textwidth}{#1}
  \begin{actionenv}#3%
    \def\insertblocktitle{#2}%
    \par%
    \usebeamertemplate{block begin}}
  {\par%
    \usebeamertemplate{block end}%
  \end{actionenv}}

\newenvironment{displaybox}[1][\textwidth]
{
    \centerline\bgroup\hfill
    \begin{beamerboxesrounded}[lower=defaultcolor,shadow=true,width=#1]{}
}
{
    \end{beamerboxesrounded}\hfill\egroup
}

\newenvironment{onlinebox}[1][4cm]
{
    \newbox\mybox
    \newdimen\myboxht
    \setbox\mybox\hbox\bgroup%
        \begin{beamerboxesrounded}[lower=defaultcolor,shadow=true,width=#1]{}
    \centering
}
{
    \end{beamerboxesrounded}\egroup
    \myboxht\ht\mybox
    \raisebox{-0.25\myboxht}{\usebox\mybox}\hspace{2pt}
}

\newenvironment{mydescription}{
    \begin{description}
        \setlength{\leftskip}{-1.5cm}}
    {\end{description}}

\newenvironment{myitemize}{
    \begin{itemize}
        \setlength{\leftskip}{-.3cm}}
    {\end{itemize}}

% footnote without a marker
\newcommand\barefootnote[1]{%
  \begingroup
  \renewcommand\thefootnote{}\footnote{#1}%
  \addtocounter{footnote}{-1}%
  \endgroup
}

% define formatting for footer
\newcommand{\myfootline}{%
    {\it
    \insertshorttitle
    \hspace*{\fill} 
    \insertshortauthor, \insertshortinstitute
    % \ifx\insertsubtitle\@empty\else, \insertshortsubtitle\fi
    \hspace*{\fill}
    \insertframenumber/\inserttotalframenumber}}

% set up footer
% \setbeamertemplate{footline}{%
%     \usebeamerfont{structure}
%     \begin{beamercolorbox}[wd=\paperwidth,ht=2.25ex,dp=1ex]{frametitle}%
%         % \Tiny\hspace*{4mm}\myfootline\hspace{4mm}
%         \tiny\hspace*{4mm}\myfootline\hspace{4mm}
%     \end{beamercolorbox}}

% remove footer
\setbeamertemplate{footline}{}

% remove navigation bar
\beamertemplatenavigationsymbolsempty

% Remove header
\setbeamertemplate{headline}[default]
% \makeatletter
%     \newenvironment{noheadline}{
%         \setbeamertemplate{headline}[default]
%         \def\beamer@entrycode{\vspace*{-\headheight}}
%     }{}
% \makeatother

\newcounter{clickerQuestionCounter}
\ifhideclicker%
\newenvironment{clickerquestion}
{ \stepcounter{clickerQuestionCounter}
  \begin{enumerate}[Q \arabic{clickerQuestionCounter}:]\color{white} }
{ \end{enumerate} }
\else%
\newenvironment{clickerquestion}
{ \stepcounter{clickerQuestionCounter}
  \begin{enumerate}[Q \arabic{clickerQuestionCounter}:] }
{ \end{enumerate} }
\fi

\ifhideclicker%
\newenvironment{clickeroptions}
{ \begin{enumerate}[\begingroup\color{white} 1)\endgroup]\color{white} }
{ \end{enumerate} }
\else%
\newenvironment{clickeroptions}
{ \begin{enumerate}[\begingroup\color{red} 1)\endgroup] }
{ \end{enumerate} }
\fi


\tikzstyle{centered} = [align=center, text centered, font=\sffamily\bfseries]
\tikzstyle{skip} = [centered, inner sep=0pt, fill]
\tikzstyle{empty} = [centered, inner sep=0pt]
\tikzstyle{inode} = [centered, circle, minimum width=4pt, fill=black, inner sep=0pt]
\tikzstyle{tnode} = [centered, circle, inner sep=1pt]
\tikzset{
  % edge styles
  level distance=10mm,
  mate/.style={edge from parent/.style={draw,distance=3pt}},
  mleft/.style={grow=left, level distance=10mm, edge from parent path={(\tikzparentnode.west)--(\tikzchildnode.east)}},
  mright/.style={grow=right, level distance=10mm, edge from parent path={(\tikzparentnode.east)--(\tikzchildnode.west)}},
  % node styles
  male/.style={rectangle,minimum size=4mm,fill=gray!80},
  female/.style={circle,minimum size=4mm,fill=gray!80},
  amale/.style={male,fill=red},
  afemale/.style={female,fill=red},
}

\newcommand{\highlight}[1]{\textcolor{violet}{\textit{\textbf{#1}}}}
\newcommand{\super}[1]{\ensuremath{^{\textrm{\sffamily #1}}}}
\newcommand{\sub}[1]{\ensuremath{_{\textrm{\sffamily #1}}}}
\newcommand{\dC}{\ensuremath{^\circ{\textrm{C}}}\xspace}
\newcommand{\degree}{\ensuremath{^\circ}\xspace}
\newcommand{\tb}{\hspace{2em}}
\providecommand{\e}[1]{\ensuremath{\times 10^{#1}}}
\newcommand{\myHangIndent}{\hangindent=5mm}

\newcommand{\spp}[1]{\textit{#1}}

\newcommand\mybullet{\leavevmode%
\usebeamertemplate{itemize item}\hspace{.5em}}

\makeatletter
\newcommand*{\rom}[1]{\expandafter\@slowromancap\romannumeral #1@}
\makeatother

\newcommand{\blankslide}{{\setbeamercolor{background canvas}{bg=black}
\setbeamercolor{whitetext}{fg=white}
\begin{frame}<handout:0>[plain]
\end{frame}}}

\newcommand{\whiteslide}{
\begin{frame}<handout:0>[plain]
\end{frame}}

\newcommand{\f}[1]{\ensuremath{F_{#1}}}
\newcommand{\x}[1]{X\ensuremath{^{#1}}}
\newcommand{\y}[1]{Y\ensuremath{^{#1}}}

% Population growth macros
\newcommand{\popsize}[1]{\ensuremath{N_{#1}}}
\newcommand{\popgrowthratediscrete}[1]{\ensuremath{\lambda_{#1}}}
\newcommand{\popgrowthrate}[1]{\ensuremath{r_{#1}}}
\newcommand{\ptime}{\ensuremath{t}\xspace}

\tikzset{hide on/.code={\only<#1>{\color{white}}}}
\tikzset{
    invisible/.style={opacity=0},
    visible on/.style={alt={#1{}{invisible}}},
    alt/.code args={<#1>#2#3}{%
        \alt<#1>{\pgfkeysalso{#2}}{\pgfkeysalso{#3}}
        % \pgfkeysalso doesn't change the path
    },
}

\bibliography{../bib/references}
% \author[J.\ Oaks]{
    %Jamie R.\ Oaks\inst{1}
    Jamie R.\ Oaks
}
% \institute[BIOL 180]{
%     \inst{}%
%         BIOL 180: Introductory Biology
% }



\title[Genes, mutation, \& alleles]{Genes, mutation, \& alleles}
% \date{\today}
% \date{April 18, 2018}
\date{}

\begin{document}

\maketitle

\begin{frame}
% \frametitle{Today's issues:}
\textbf{At the molecular level, what is a gene? An allele? A mutation?} \\
% \vspace{5mm}
% \tableofcontents[subsectionstyle=hide]
\end{frame}

\section{The molecular nature of the gene}

\begin{frame}[t]

    {\Large What is the definition of a gene?}\footnote{\tiny It's complicated.}

    \uncover<2->{
    \begin{itemize}
        \item A section of DNA that influences the phenotype for a particular
            trait. 
    \end{itemize}
    }

    % \nbox{A section of DNA that influences the phenotype for a particular
    %     trait.}
\end{frame}

\tikzstyle{nuc} = [
    minimum width=2.3ex, minimum height=2.3ex, node distance=2.3ex]
\tikzstyle{space} = [
    minimum width=2.3ex, minimum height=1ex, node distance=1ex]

% \begin{frame}[t]
%     \frametitle{What are genes made of?}
%     \begin{adjustwidth}{-1.5em}{-1.5em}
%     \vspace{-2mm}
%     \centering{
%         \includegraphics[width=0.5\linewidth]{dna-crop.png}
%     }
%     \begin{center}
%     \begin{tikzpicture}[font=\sffamily]
%         \node[visible on=<1->, name=tn1, nuc] {A};
%         \node[visible on=<1->, name=tn2, nuc, right of=tn1] {T};
%         \node[visible on=<1->, name=tn3, nuc, right of=tn2] {G};
%         \node[visible on=<1->, name=tn4, nuc, right of=tn3] {T};
%         \node[visible on=<1->, name=tn5, nuc, right of=tn4] {T};
%         \node[visible on=<1->, name=tn6, nuc, right of=tn5] {A};
%         \node[visible on=<1->, name=tn7, nuc, right of=tn6] {A};
%         \node[visible on=<1->, name=tn8, nuc, right of=tn7] {G};
%         \node[visible on=<1->, name=tn9, nuc, right of=tn8] {T};
%         \node[visible on=<1->, name=tn10, nuc, right of=tn9] {G};
%         \node[visible on=<1->, name=tn11, nuc, right of=tn10] {A};
%         \node[visible on=<1->, name=tn12, nuc, right of=tn11] {G};
%         \node[visible on=<1->, name=tn13, nuc, right of=tn12] {G};
%         \node[visible on=<1->, name=tn14, nuc, right of=tn13] {C};
%         \node[visible on=<1->, name=tn15, nuc, right of=tn14] {T};
%         \node[visible on=<1->, name=tn16, nuc, right of=tn15] {A};
%         \node[visible on=<1->, name=tn17, nuc, right of=tn16] {A};
%         \node[visible on=<1->, name=tn18, nuc, right of=tn17] {T};
%         \node[visible on=<1->, name=tn19, nuc, right of=tn18] {A};
%         \node[visible on=<1->, name=tn20, nuc, right of=tn19] {G};

%         \node[visible on=<1->, name=sn1, nuc, below of=tn1] {};
%         \node[visible on=<1->, name=sn2, nuc, below of=tn2] {};
%         \node[visible on=<1->, name=sn3, nuc, below of=tn3] {};
%         \node[visible on=<1->, name=sn4, nuc, below of=tn4] {};
%         \node[visible on=<1->, name=sn5, nuc, below of=tn5] {};
%         \node[visible on=<1->, name=sn6, nuc, below of=tn6] {};
%         \node[visible on=<1->, name=sn7, nuc, below of=tn7] {};
%         \node[visible on=<1->, name=sn8, nuc, below of=tn8] {};
%         \node[visible on=<1->, name=sn9, nuc, below of=tn9] {};
%         \node[visible on=<1->, name=sn10, nuc, below of=tn10] {};
%         \node[visible on=<1->, name=sn11, nuc, below of=tn11] {};
%         \node[visible on=<1->, name=sn12, nuc, below of=tn12] {};
%         \node[visible on=<1->, name=sn13, nuc, below of=tn13] {};
%         \node[visible on=<1->, name=sn14, nuc, below of=tn14] {};
%         \node[visible on=<1->, name=sn15, nuc, below of=tn15] {};
%         \node[visible on=<1->, name=sn16, nuc, below of=tn16] {};
%         \node[visible on=<1->, name=sn17, nuc, below of=tn17] {};
%         \node[visible on=<1->, name=sn18, nuc, below of=tn18] {};
%         \node[visible on=<1->, name=sn19, nuc, below of=tn19] {};
%         \node[visible on=<1->, name=sn20, nuc, below of=tn20] {};

%         \node[visible on=<1->, name=bn1, nuc, below of=sn1] {T};
%         \node[visible on=<1->, name=bn2, nuc, below of=sn2] {A};
%         \node[visible on=<1->, name=bn3, nuc, below of=sn3] {C};
%         \node[visible on=<1->, name=bn4, nuc, below of=sn4] {A};
%         \node[visible on=<1->, name=bn5, nuc, below of=sn5] {A};
%         \node[visible on=<1->, name=bn6, nuc, below of=sn6] {T};
%         \node[visible on=<1->, name=bn7, nuc, below of=sn7] {T};
%         \node[visible on=<1->, name=bn8, nuc, below of=sn8] {C};
%         \node[visible on=<1->, name=bn9, nuc, below of=sn9] {A};
%         \node[visible on=<1->, name=bn10, nuc, below of=sn10] {C};
%         \node[visible on=<1->, name=bn11, nuc, below of=sn11] {T};
%         \node[visible on=<1->, name=bn12, nuc, below of=sn12] {C};
%         \node[visible on=<1->, name=bn13, nuc, below of=sn13] {C};
%         \node[visible on=<1->, name=bn14, nuc, below of=sn14] {G};
%         \node[visible on=<1->, name=bn15, nuc, below of=sn15] {A};
%         \node[visible on=<1->, name=bn16, nuc, below of=sn16] {T};
%         \node[visible on=<1->, name=bn17, nuc, below of=sn17] {T};
%         \node[visible on=<1->, name=bn18, nuc, below of=sn18] {A};
%         \node[visible on=<1->, name=bn19, nuc, below of=sn19] {T};
%         \node[visible on=<1->, name=bn20, nuc, below of=sn20] {C};

%         \path[-] (tn1) edge [visible on=<1->, thick] (bn1);
%         \path[-] (tn2) edge [visible on=<1->, thick] (bn2);
%         \path[-] (tn3) edge [visible on=<1->, thick] (bn3);
%         \path[-] (tn4) edge [visible on=<1->, thick] (bn4);
%         \path[-] (tn5) edge [visible on=<1->, thick] (bn5);
%         \path[-] (tn6) edge [visible on=<1->, thick] (bn6);
%         \path[-] (tn7) edge [visible on=<1->, thick] (bn7);
%         \path[-] (tn8) edge [visible on=<1->, thick] (bn8);
%         \path[-] (tn9) edge [visible on=<1->, thick] (bn9);
%         \path[-] (tn10) edge [visible on=<1->, thick] (bn10);
%         \path[-] (tn11) edge [visible on=<1->, thick] (bn11);
%         \path[-] (tn12) edge [visible on=<1->, thick] (bn12);
%         \path[-] (tn13) edge [visible on=<1->, thick] (bn13);
%         \path[-] (tn14) edge [visible on=<1->, thick] (bn14);
%         \path[-] (tn15) edge [visible on=<1->, thick] (bn15);
%         \path[-] (tn16) edge [visible on=<1->, thick] (bn16);
%         \path[-] (tn17) edge [visible on=<1->, thick] (bn17);
%         \path[-] (tn18) edge [visible on=<1->, thick] (bn18);
%         \path[-] (tn19) edge [visible on=<1->, thick] (bn19);
%         \path[-] (tn20) edge [visible on=<1->, thick] (bn20);
%     \end{tikzpicture}
%     \end{center}
%     \vspace{-2mm}
%     \begin{itemize}
%         \item What do the ``bars'' in the cartoon, and the letters below it
%             represent?
%             \nbox{\small The bars and letters both represent nucleotides (or
%                 bases)---the building block of DNA}

%         \item Do you notice any patterns in the 2 rows of letters?
%             \nbox{\small A's pair with T's; G's pair with C's (= complementary base
%                 pairing)}

%         \item What does a chromosome consist of?
%             \nbox{\small One long DNA molecule (a ``double helix'') (in some species,
%                 like eukaryotes, it is wrapped around proteins)}
%     \end{itemize}

%     \end{adjustwidth}
% \end{frame}

% \begin{frame}[t]
%     \begin{adjustwidth}{-1.5em}{-1.5em}
%     Why is it significant that in some parts of some genes, groups of 3 bases
%     code for an amino acid?

%     \vspace{5mm}
%     \centering{
%         \includegraphics[width=\linewidth]{codons.png}
%     }

%     \nbox{Some genes code for proteins. Amino acids are the building blocks of
%         proteins, which form structures and act as ``machines'' of the cell.
%         DNA stores the information (code) for building and running cells.}
%     \end{adjustwidth}
% \end{frame}

\begin{frame}[t]
    \begin{adjustwidth}{-1.5em}{-1.5em}
    Different sections of a gene are referred to as \highlight{structural} or
    \highlight{regulatory}. What's the difference?

    \vspace{5mm}
    \centering{
        \includegraphics[width=\linewidth]{gene-regions.png}
    }

    \nbox{Structural regions code for an RNA or protein that functions in the
        cell; regulatory regions are responsible for controlling the
        \highlight{expression} of the gene (i.e., on/off/more/less).}
    \end{adjustwidth}
\end{frame}

\begin{frame}[t]
    \frametitle{The importance of regulatory sequences}
    \begin{adjustwidth}{-1.5em}{-1.5em}
    \begin{itemize}
        \item Muscle cells and nerve cells contain the same chromosomes. Why
            are the cells so different?

            \nbox{They express different genes at different times and in
                different quantities; this leads to different structure and
                function}

        \item Dr.\ Oaks has the genes required to make a uterus. Why doesn't he
            have one?

            \nbox{The genes responsible for the uterus never got turned on}

        \item In many cases, the proteins produced by homologous genes in
            chimps and humans are identical or nearly identical. Why are the
            two species so different?

            \nbox{There are differences in the regulatory sequences so that the
                timing and amount of expression of the proteins is different}
    \end{itemize}
    \end{adjustwidth}
\end{frame}

\section{The central dogma of molecular biology}

\begin{frame}
    % \frametitle{The central dogma of molecular biology}
    \begin{adjustwidth}{-1.5em}{-1.5em}
    \begin{columns}
        \column{0.45\linewidth}
        
        \includegraphics[width=\textwidth]{central-dogma.png}

        \column{0.55\linewidth}

        \uncover<2->{
        \begin{itemize}
            \item Arrows represent enzyme-catalyzed reactions
            % \item What do these arrows represent?
                \nbox{\tiny Enzyme-catalyzed reactions: Chemical reactions are making
                    new molecules from information coded in the DNA/RNA, and
                    these reactions are catalyzed by enzymes.}
            
            \item What makes up the genotype?

                \nbox{The sequence of bases in DNA; different sequences are
                    different alleles}

            \item What makes up the phenotype?

                \nbox{RNA and protein products}
        \end{itemize}
        }

    \end{columns}
    \end{adjustwidth}
\end{frame}

\begin{frame}[t]
    \begin{adjustwidth}{-1.5em}{-1.5em}
        \begin{itemize}
            \item At the molecular level, what is the ``gene for flower
                color?''

                \nbox{A DNA sequence (section of a chromosome) that codes for a
                    product that affects flower color.}

                \vspace{8mm}
            \item At the molecular level, what is an allele?

                \nbox{Any version of a gene that differs in DNA sequence.}

                \vspace{1.2cm}
            \item Often in the media you will hear, ``scientists discovered the
                gene for \underline{\ \ \ \ \ }.'' What is wrong with this?

                \nbox{It's almost always a new \highlight{allele} that is
                    found.}
        \end{itemize}
    \end{adjustwidth}
\end{frame}


\section{The molecular nature of mutation}

\begin{frame}[t]
    \begin{adjustwidth}{-1.5em}{-1.5em}
        \begin{itemize}
            \item When chromosomes replicate, DNA is copied by enzymes

            \vspace{5mm}
            \item The enzymes make mistakes, at random
                \begin{itemize}
                    \item What does random mean?

                        \nbox{During replication mistakes are random with
                            respect to location (which base gets changed), type
                            (which base it changes to), and fitness!}
                \end{itemize}

            \vspace{5mm}
            \item What are the consequences of these mistakes, for the genotype?

                \nbox{It changes the genotype by creating a new allele}

            \vspace{5mm}
            \item What are the consequences of these mistakes, for the phenotype?

                \nbox{The phenotype will change \highlight{if} the gene's
                    product, or expression of the product, is changed.}

        \end{itemize}

        \vspace{1cm}
        \uncover<2->{In humans, the average gamete has 0.1--1 mutations (out of
            3.4 billion bases)}
    \end{adjustwidth}
\end{frame}

% \begin{frame}[t]
%     \begin{adjustwidth}{-1.5em}{-1.5em}
%         \begin{itemize}
%             \item In humans, an error occurs every 2,000,000,000 bases
%             \item Our haploid genome is 3,400,000,000 nucleotides
%             \item So, the average gamete has about 34 mutations
%         \end{itemize}
%     \end{adjustwidth}
% \end{frame}

% \begin{frame}[t]
%     \begin{adjustwidth}{-1.5em}{-1.5em}
%         \begin{itemize}
%             \item What happens to the genotype and phenotype if there is a
%                 change in the sequence of bases in the \highlight{coding
%                     region} of a gene?

%                 \nbox{Genotype changes, because now there is a new allele.
%                     Phenotype may change, \highlight{if} the changed sequence
%                     affects the structure or function of the gene product.}

%                 \vspace{8mm}
%             \item What happens to the genotype and phenotype if there is a
%                 change in the sequence of bases in the \highlight{regulatory
%                     regions} of a gene?

%                 \nbox{Genotype changes, because now there is a new allele.
%                     Phenotype may change, \highlight{if} the changed sequence
%                     affects the expression of the gene product.}

%         \end{itemize}
%     \end{adjustwidth}
% \end{frame}


% \begin{frame}[t]
%     \begin{adjustwidth}{-1.5em}{-1.5em}
%         \begin{itemize}
%             \item Some copying errors not only change the DNA base sequence,
%                 but also the RNA or protein product

%                 \begin{itemize}
%                     \item How do these errors affect the phenotype?

%                         \nbox{It will cause a change, if the new RNA or protein
%                             functions differently}

%                     \vspace{2mm}
%                     \item Doe these errors create new alleles?

%                         \nbox{Yes}
%                 \end{itemize}

%             \vspace{2mm}
%             \item Some copying errors change the DNA base sequence, but do
%                 \highlight{NOT} change the protein product

%                 \begin{itemize}
%                     \item How do these errors affect the phenotype?

%                         \nbox{If the error occurs in a regulatory region of a
%                             gene, and changes the expression of the product
%                             (when/where/how much), it can change the phenotype
%                             (otherwise, no change)}

%                     \vspace{2mm}
%                     \item Do these errors create new alleles?

%                         \nbox{Yes. It is still a unique version of the gene.}
%                 \end{itemize}

%         \end{itemize}
%     \end{adjustwidth}
% \end{frame}

% \begin{frame}[t]
%     \begin{adjustwidth}{-1.5em}{-1.5em}
%         \begin{itemize}
%             \item Mistakes in meiosis can lead to a doubling of the number of
%                 chromosomes or other changes in chromosome number (e.g., if the
%                 homologs fail to pull apart).

%                 \begin{itemize}
%                     \item Do these errors affect the genotype?

%                         \nbox{Yes, it will change the genotype of every gene on
%                             the chromosomes in question. There are now more
%                             copies of each gene.}

%                     \vspace{7mm}
%                     \item Do these errors affect the phenotype?

%                         \nbox{Maybe; for some genes there can be differences in
%                             expression}
%                 \end{itemize}

%         \end{itemize}
%         \vspace{5mm}
%         \begin{flushright}
%             \includegraphics[height=0.28\textheight]{strawberry-ploidy.png}
%         \end{flushright}
%     \end{adjustwidth}
% \end{frame}

% \begin{frame}[t]
%     \begin{adjustwidth}{-1.5em}{-1.5em}
%         \begin{itemize}
%             \item Damage from ultraviolet radiation or toxic chemicals can
%                 break chromosomes.

%                 \begin{itemize}
%                     \item Do these breaks affect the genotype?

%                         \nbox{It depends on where the break occurs. If the
%                             break occurs within a gene (coding or regulatory
%                             regions), then yes it will. It will result in a
%                             new, broken (not functional) allele}

%                         \vspace{1cm}
%                     \item Do these breaks affect the phenotype?

%                         \nbox{It depends on where the break occurs. If the
%                             break occurs within a gene (coding or regulatory
%                             regions), then it probably will. The gene product
%                             will no longer be produced}
%                 \end{itemize}

%         \end{itemize}
%     \end{adjustwidth}
% \end{frame}

% \clickerslide{
% \begin{frame}
%     \begin{clickerquestion}
%         \item Mutation is a random process, at the physical level. At the
%             evolutionary level, it is random with respect to what?
%         \begin{clickeroptions}
%             \item Gender
%             \item Cell type
%             \item Developmental stage
%             \item \clickeranswer{Fitness}
%         \end{clickeroptions}
%     \end{clickerquestion}
% \end{frame}
% }

% \clickerslide{
% \begin{frame}
%     \begin{clickerquestion}
%         \item Why is mutation constant?
%         \begin{clickeroptions}
%             \item \clickeranswer{It can happen in males or females, in any type
%                     of cell, and during mitosis or meiosis.}
%             \item Mutation rates can increase under certain environmental
%                 conditions (e.g., presence of carcinogens, high UV light).
%             \item Because evolution is random.
%             \item The errors that occur in meiosis and mitosis are not directed
%                 with respect to fitness.
%         \end{clickeroptions}
%     \end{clickerquestion}
% \end{frame}
% }

\clickerslide{
\begin{frame}
    \begin{clickerquestion}
        \item  How does mutation create \highlight{heritable} variation?
        \begin{clickeroptions}
            \item Because mutation can occur during either mitosis or
                meiosis---anytime that DNA is copied.
            \item Because they are random, most changes in DNA are not
                advantageous.
            \item Because otherwise, evolution would ``grind to a halt'' (there
                would be no genetic variation) 
            \item \clickeranswer{Changes in DNA that occur prior to meiosis (in
                    germ cells) are passed on to the next generation.}
        \end{clickeroptions}
    \end{clickerquestion}
\end{frame}
}

\begin{frame}[t]
    \begin{adjustwidth}{-1.5em}{-1.5em}
        \begin{itemize}
            \item Key definition: A \highlight{mutation} is any change in an
                organism's DNA.

                \begin{itemize}
                    \item Why doesn't Lamarckian inheritance (acquired
                        characters) work?

                        \nbox{Two reasons: (1) Changes in phenotype during an
                            individual's lifetime is almost always due to
                            changes in gene expression, NOT a new mutation! (2)
                            Even if a mutation occurred in an individual and
                            changed its phenotype (e.g., cancer), it couldn't
                            be passed on because the mutation is not in the
                            germ line!}

                \end{itemize}

        \end{itemize}
    \end{adjustwidth}
\end{frame}

\section{Four sources of genetic variation}

\subsection{Mutation}

\begin{frame}[t]
    \frametitle{Four sources of genetic variation}
    \begin{enumerate}
        \item \textbf{Mutation}
        \item \textbf{Independent assortment}
        \item \textbf{Recombination}
        \item \textbf{Outcrossing (sex)}
    \end{enumerate}

    Mutation is the ultimate source of variation; the other 3 create new
    combinations of alleles
\end{frame}

% \begin{frame}[t]
%     \frametitle{Four sources of genetic variation}
%     \begin{adjustwidth}{-3em}{-1.5em}
%         \vspace{-4mm}
%         \begin{itemize}
%             \item[1.]<2-> \textbf{Mutation:} Based on 2010 sequencing data from humans,
%                 an average gamete contains approximately 1 base-substitution
%                 mutation in every $10^8$ bases.

%                 \begin{itemize}
%                     \item<3-> How many mutations is this per gamete? {\scriptsize(Haploid
%                         genome = 3.4\e{9} bases)}

%                         \nbox{\small $\frac{1\textrm{mutation}}{10^8\textrm{bases}}
%                             \times 3.4\e{9}\textrm{bases} \approx 34 \textrm{mutations}$}

%                     \item<3-> How do these mutations create genetic variation?

%                         \nbox{They create new alleles, and possibly new
%                             phenotypes}

%                     \item<3-> When and where do these mutation occur?

%                         \nbox{During chromosome replication prior to meiosis in
%                             gametocytes}

%                     \item<3-> What is the physical cause?

%                         \nbox{\footnotesize Errors by enzymes that copy DNA}
%                 \end{itemize}
%                 \nbox{{\scriptsize Ultimate source of variation; sources 2--4 create new
%                     combos of alleles introduced via mutation}}
%         \end{itemize}

%     \end{adjustwidth}
%     \barefootnote{\shortfullcite{Lynch2010}}
% \end{frame}

\subsection{Independent assortment}

\begin{frame}[t]
    % \frametitle{Four sources of genetic variation}
    \begin{adjustwidth}{-1.5em}{-1.5em}
        \begin{itemize}
            \item[2.] \textbf{Independent assortment:}

                \begin{itemize}
                    \item How does it create genetic variation?

                        \nbox{Each gamete gets a random assortment of maternal
                            and paternal chromosomes (variation AMONG
                            chromosomes). A human can create $2^{23} =
                            8.39\e{6}$ unique gametes via independent
                            assortment!}

                    \item When does it occur?

                        \nbox{Metaphase of meiosis I}

                    \vspace{9mm}
                    \item What is the physical cause?

                        \nbox{Synapsed homologs line up randomly with respect to each
                            other}
                \end{itemize}
        \end{itemize}

    \end{adjustwidth}
\end{frame}

\subsection{Recombination}

\begin{frame}[t]
    % \frametitle{Four sources of genetic variation}
    \begin{adjustwidth}{-1.5em}{-1.5em}
        \begin{itemize}
            \item[3.] \textbf{Recombination (crossing over):}

                \begin{itemize}
                    \item How does it create genetic variation?

                        \nbox{Crossing over creates random changes in which
                            alleles are on each homologous chromosome---end up
                            with mix of maternal and paternal sections on the
                            same chromosome! (variation WITHIN chromosomes).
                            In outcrossing (sexual) species, recombination can
                            create \emph{almost} an infinite number of unique
                            gametes!}

                    \item When does it occur?

                        \nbox{Prophase of meiosis I}

                    \vspace{9mm}
                    \item What is the physical cause?

                        \nbox{Crossing over (exchanging sections) between
                            non-sister chromatids of synapsed homologs}
                \end{itemize}
        \end{itemize}

    \end{adjustwidth}
\end{frame}

\subsection{Outcrossing}

\begin{frame}[t]
    % \frametitle{Four sources of genetic variation}
    \begin{adjustwidth}{-1.5em}{-1.5em}
        \begin{itemize}
            \item[4.] \textbf{Outcrossing (sex):}

                \begin{itemize}
                    \item How does it create genetic variation?

                        \nbox{Combinations of chromosomes from different
                            individuals creates new combinations of alleles.}

                        \vspace{3mm}
                    \item When does it occur?

                        \nbox{Sexual reproduction (fertilization)}

                    \vspace{8mm}
                    \item What is the physical cause?

                        \nbox{Combination of haploid gametes to create a
                            diploid (or higher ploidy)}
                \end{itemize}
        \end{itemize}

        \vspace{1.5cm}
        Sex gives independent assortment and crossing over more variation to
        shuffle---tons of variation!
    \end{adjustwidth}
\end{frame}


\section{The evolution of sex}

\begin{frame}[t]
    \frametitle{The evolution of sex}
    \begin{adjustwidth}{-1.5em}{-1.5em}
        \begin{itemize}
            \item Why go to all this trouble, just to produce offspring that
                are genetically different from each other and from their
                parents?

                \uncover<2->{
                \vspace{1mm}
                Hypotheses:
                \begin{enumerate}
                    \item Purifying selection (getting rid of deleterious alleles)
                        \nbox{Thanks to sex, some offspring will not have the
                            deleterious (low-fitness) alleles of their
                            parent(s); not possible without sex! E.g., if you have a
                            deleterious allele and can only clone yourself, all
                            of your offspring will have that deleterious
                            allele!}

                        \vspace{1mm}
                    \item Changing environment (producing high-fitness phenotypes)

                        \nbox{Sex increases the probability that at least some
                            of offspring will have higher fitness in a changed
                            environment.}
                    
                    \begin{itemize}
                        \item What aspects of the environment change so fast
                            that sex is advantageous compared to cloning?

                        \nbox{Disease!}
                    \end{itemize}

                \end{enumerate}
                }
        \end{itemize}

    \end{adjustwidth}
\end{frame}

\begin{frame}[t]
    \begin{adjustwidth}{-1.5em}{-1.5em}
        \vspace{-3mm}
        Morran et al.\ did an experiment with a species of roundworms that
        outcrosses 20\% of the time on average. They tracked the evolution of a
        population in the presence and absence of a bacterial parasite. What is
        the take-home message here?

        \centering{
            \includegraphics[height=0.6\textheight]{roundworm-outcrossing-plot.png}
        }

        \nbox{\tiny In presence of pathogen, worms that outcross more
            often have higher fitness---this supports the changing-environment
            hypothesis.}
        \vspace{-2mm}
        \barefootnote{\shortfullcite{Morran2011}}
    \end{adjustwidth}
    \note[item]{Roundworms are hermaphroditic; they can self or outcross; this
        trait has heritable variation---it can evolve!}
\end{frame}

% \begin{frame}[t]
%     \begin{adjustwidth}{-1.5em}{-1.5em}
%         Why do college students think that t-shirts smell nice, \highlight{IF}
%         they were worn by someone with different MHC alleles than they have?

%         \begin{itemize}
%             \item MHC = Major Histocompatibility Complex: Proteins on surfaces
%                 of cells that are \highlight{critical} for the immune system.
%                 They ``recognize'' pathogens, and ``flag them.''
%         \end{itemize}
%         \barefootnote{\shortfullcite{Wedekind1995}}
%     \end{adjustwidth}
% \end{frame}


\end{document}


\clickerslide{
\begin{frame}
    \begin{clickerquestion}
        \item 
        \begin{clickeroptions}
            \item 
            \item 
            \item 
            \item 
        \end{clickeroptions}
    \end{clickerquestion}
\end{frame}
}
