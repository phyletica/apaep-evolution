\documentclass[14pt,table]{beamer}
% \documentclass[14pt,table,handout]{beamer}
% \documentclass[table]{beamer}
% \documentclass[table,handout]{beamer}
% \setbeameroption{show notes}
% \setbeameroption{hide notes}
% \setbeameroption{show only notes}
\usepackage{varwidth}

\newif\ifhide
\newif\ifpost
\newif\ifhideclicker

\hidetrue
% \hideclickertrue
% \posttrue

\newcommand{\whiteout}[1]{\textcolor{white}{#1}}
% \newcommand{\whiteoutbox}[1]{\fcolorbox{white}{white}{\parbox{\dimexpr \linewidth-2\fboxsep-2\fboxrule}{\whiteout{#1}}}}
% \newcommand{\notebox}[1]{\fcolorbox{blue}{white}{\parbox{\dimexpr \linewidth-2\fboxsep-2\fboxrule}{#1}}}
\newcommand{\whiteoutbox}[1]{\fcolorbox{white}{white}{\parbox{\linewidth}{\whiteout{#1}}}}
\newcommand{\notebox}[1]{\fcolorbox{blue}{white}{\parbox{\linewidth}{#1}}}
\newcommand{\blankbox}[1]{\phantom{\varwidth{\linewidth}\whiteoutbox{#1}\endvarwidth}}
\newcommand{\blank}[1]{\phantom{\varwidth{\linewidth}#1\endvarwidth}}

\ifhide%
    \newcommand{\hmask}[1]{\blank{#1}}%
\else%
    \newcommand{\hmask}[1]{#1}%
\fi

\ifhide%
    \newcommand{\cmask}[1]{\blank{#1}}%
\else%
    \newcommand{\cmask}[1]{\textcolor{blue}{#1}}%
\fi

\ifhide%
    \newcommand{\wout}[1]{\whiteout{#1}}%
\else%
    \newcommand{\wout}[1]{#1}%
\fi

\ifhide%
    \newcommand{\hignore}[1]{}%
\else%
    \newcommand{\hignore}[1]{#1}%
\fi

\ifpost%
    \newcommand{\nopost}[1]{}%
\else%
    \newcommand{\nopost}[1]{#1}%
\fi

\ifhideclicker%
    \newcommand{\clickerslide}[1]{\stepcounter{clickerQuestionCounter}%
        \begin{frame}[t]
            \textcolor{blue}{Q \arabic{clickerQuestionCounter}:}
        \end{frame}}
\else%
    \newcommand{\clickerslide}[1]{#1}%
\fi

\ifhideclicker%
    \newcommand{\clickerpost}[1]{\stepcounter{clickerQuestionCounter}%
        \begin{frame}[t]
            \textcolor{blue}{Q \arabic{clickerQuestionCounter}:}
        \end{frame}}
\else%
    \newcommand{\clickerpost}[1]{\stepcounter{clickerQuestionCounter}#1}%
\fi

\ifhide%
    \newcommand{\hidebox}[1]{\blank{#1}}%
\else%
    \newcommand{\hidebox}[1]{\notebox{#1}}%
\fi

\ifhide%
    \newcommand{\wbox}[1]{\whiteoutbox{#1}}%
\else%
    \newcommand{\wbox}[1]{\notebox{#1}}%
\fi

\ifhide%
    \newcommand{\nbox}[1]{\blankbox{#1}}%
\else%
    \newcommand{\nbox}[1]{\notebox{#1}}%
\fi

\ifhideclicker%
    \newcommand{\clickeranswer}[1]{#1}%
\else%
    \ifhide%
        \newcommand{\clickeranswer}[1]{#1}%
    \else%
        \newcommand{\clickeranswer}[1]{\textbf{\textcolor{blue}{#1}}}%
    \fi
\fi

\usepackage{beamerthemesplit}
% \usetheme{boxes}
\usetheme{Malmoe}
\usecolortheme{seahorse}
% \usecolortheme{seagull}
\usepackage{ifthen}
\usepackage{xspace}
\usepackage{multirow}
\usepackage{multicol}
\usepackage{booktabs}
\usepackage{xcolor}
\usepackage{wasysym}
\usepackage{comment}
\usepackage{hyperref}
\hypersetup{pdfborder={0 0 0}, colorlinks=true, urlcolor=blue, linkcolor=blue, citecolor=blue}
\usepackage{changepage}
\usepackage[compatibility=false]{caption}
\captionsetup[figure]{font=scriptsize, labelformat=empty, textformat=simple, justification=centering, skip=2pt}
\usepackage{tikz}
\usepackage{pgf}
\usetikzlibrary{trees,calc,backgrounds,arrows,positioning,automata}

\usepackage{array}
\newcolumntype{L}[1]{>{\raggedright\let\newline\\\arraybackslash\hspace{0pt}}p{#1}}
\newcolumntype{C}[1]{>{\centering\let\newline\\\arraybackslash\hspace{0pt}}p{#1}}
\newcolumntype{R}[1]{>{\raggedleft\let\newline\\\arraybackslash\hspace{0pt}}p{#1}}

\usepackage[bibstyle=joaks-slides,maxcitenames=1,mincitenames=1,backend=biber]{biblatex}

\newrobustcmd*{\shortfullcite}{\AtNextCite{\renewbibmacro{title}{}\renewbibmacro{in:}{}\renewbibmacro{number}{}}\fullcite}

\newrobustcmd*{\footlessfullcite}{\AtNextCite{\renewbibmacro{title}{}\renewbibmacro{in:}{}}\footfullcite}

% Make all footnotes smaller
% \renewcommand{\footnotesize}{\scriptsize}

\definecolor{myGray}{gray}{0.9}
\colorlet{rowred}{red!30!white}

\setbeamertemplate{blocks}[rounded][shadow=true]

\setbeamercolor{defaultcolor}{bg=structure!30!normal text.bg,fg=black}
\setbeamercolor{block body}{bg=structure!30!normal text.bg,fg=black}
\setbeamercolor{block title}{bg=structure!50!normal text.bg,fg=black}

\newenvironment<>{varblock}[2][\textwidth]{%
  \setlength{\textwidth}{#1}
  \begin{actionenv}#3%
    \def\insertblocktitle{#2}%
    \par%
    \usebeamertemplate{block begin}}
  {\par%
    \usebeamertemplate{block end}%
  \end{actionenv}}

\newenvironment{displaybox}[1][\textwidth]
{
    \centerline\bgroup\hfill
    \begin{beamerboxesrounded}[lower=defaultcolor,shadow=true,width=#1]{}
}
{
    \end{beamerboxesrounded}\hfill\egroup
}

\newenvironment{onlinebox}[1][4cm]
{
    \newbox\mybox
    \newdimen\myboxht
    \setbox\mybox\hbox\bgroup%
        \begin{beamerboxesrounded}[lower=defaultcolor,shadow=true,width=#1]{}
    \centering
}
{
    \end{beamerboxesrounded}\egroup
    \myboxht\ht\mybox
    \raisebox{-0.25\myboxht}{\usebox\mybox}\hspace{2pt}
}

\newenvironment{mydescription}{
    \begin{description}
        \setlength{\leftskip}{-1.5cm}}
    {\end{description}}

\newenvironment{myitemize}{
    \begin{itemize}
        \setlength{\leftskip}{-.3cm}}
    {\end{itemize}}

% footnote without a marker
\newcommand\barefootnote[1]{%
  \begingroup
  \renewcommand\thefootnote{}\footnote{#1}%
  \addtocounter{footnote}{-1}%
  \endgroup
}

% define formatting for footer
\newcommand{\myfootline}{%
    {\it
    \insertshorttitle
    \hspace*{\fill} 
    \insertshortauthor, \insertshortinstitute
    % \ifx\insertsubtitle\@empty\else, \insertshortsubtitle\fi
    \hspace*{\fill}
    \insertframenumber/\inserttotalframenumber}}

% set up footer
% \setbeamertemplate{footline}{%
%     \usebeamerfont{structure}
%     \begin{beamercolorbox}[wd=\paperwidth,ht=2.25ex,dp=1ex]{frametitle}%
%         % \Tiny\hspace*{4mm}\myfootline\hspace{4mm}
%         \tiny\hspace*{4mm}\myfootline\hspace{4mm}
%     \end{beamercolorbox}}

% remove footer
\setbeamertemplate{footline}{}

% remove navigation bar
\beamertemplatenavigationsymbolsempty

% Remove header
\setbeamertemplate{headline}[default]
% \makeatletter
%     \newenvironment{noheadline}{
%         \setbeamertemplate{headline}[default]
%         \def\beamer@entrycode{\vspace*{-\headheight}}
%     }{}
% \makeatother

\newcounter{clickerQuestionCounter}
\ifhideclicker%
\newenvironment{clickerquestion}
{ \stepcounter{clickerQuestionCounter}
  \begin{enumerate}[Q \arabic{clickerQuestionCounter}:]\color{white} }
{ \end{enumerate} }
\else%
\newenvironment{clickerquestion}
{ \stepcounter{clickerQuestionCounter}
  \begin{enumerate}[Q \arabic{clickerQuestionCounter}:] }
{ \end{enumerate} }
\fi

\ifhideclicker%
\newenvironment{clickeroptions}
{ \begin{enumerate}[\begingroup\color{white} 1)\endgroup]\color{white} }
{ \end{enumerate} }
\else%
\newenvironment{clickeroptions}
{ \begin{enumerate}[\begingroup\color{red} 1)\endgroup] }
{ \end{enumerate} }
\fi


\tikzstyle{centered} = [align=center, text centered, font=\sffamily\bfseries]
\tikzstyle{skip} = [centered, inner sep=0pt, fill]
\tikzstyle{empty} = [centered, inner sep=0pt]
\tikzstyle{inode} = [centered, circle, minimum width=4pt, fill=black, inner sep=0pt]
\tikzstyle{tnode} = [centered, circle, inner sep=1pt]
\tikzset{
  % edge styles
  level distance=10mm,
  mate/.style={edge from parent/.style={draw,distance=3pt}},
  mleft/.style={grow=left, level distance=10mm, edge from parent path={(\tikzparentnode.west)--(\tikzchildnode.east)}},
  mright/.style={grow=right, level distance=10mm, edge from parent path={(\tikzparentnode.east)--(\tikzchildnode.west)}},
  % node styles
  male/.style={rectangle,minimum size=4mm,fill=gray!80},
  female/.style={circle,minimum size=4mm,fill=gray!80},
  amale/.style={male,fill=red},
  afemale/.style={female,fill=red},
}

\newcommand{\highlight}[1]{\textcolor{violet}{\textit{\textbf{#1}}}}
\newcommand{\super}[1]{\ensuremath{^{\textrm{\sffamily #1}}}}
\newcommand{\sub}[1]{\ensuremath{_{\textrm{\sffamily #1}}}}
\newcommand{\dC}{\ensuremath{^\circ{\textrm{C}}}\xspace}
\newcommand{\degree}{\ensuremath{^\circ}\xspace}
\newcommand{\tb}{\hspace{2em}}
\providecommand{\e}[1]{\ensuremath{\times 10^{#1}}}
\newcommand{\myHangIndent}{\hangindent=5mm}

\newcommand{\spp}[1]{\textit{#1}}

\newcommand\mybullet{\leavevmode%
\usebeamertemplate{itemize item}\hspace{.5em}}

\makeatletter
\newcommand*{\rom}[1]{\expandafter\@slowromancap\romannumeral #1@}
\makeatother

\newcommand{\blankslide}{{\setbeamercolor{background canvas}{bg=black}
\setbeamercolor{whitetext}{fg=white}
\begin{frame}<handout:0>[plain]
\end{frame}}}

\newcommand{\whiteslide}{
\begin{frame}<handout:0>[plain]
\end{frame}}

\newcommand{\f}[1]{\ensuremath{F_{#1}}}
\newcommand{\x}[1]{X\ensuremath{^{#1}}}
\newcommand{\y}[1]{Y\ensuremath{^{#1}}}

% Population growth macros
\newcommand{\popsize}[1]{\ensuremath{N_{#1}}}
\newcommand{\popgrowthratediscrete}[1]{\ensuremath{\lambda_{#1}}}
\newcommand{\popgrowthrate}[1]{\ensuremath{r_{#1}}}
\newcommand{\ptime}{\ensuremath{t}\xspace}

\tikzset{hide on/.code={\only<#1>{\color{white}}}}
\tikzset{
    invisible/.style={opacity=0},
    visible on/.style={alt={#1{}{invisible}}},
    alt/.code args={<#1>#2#3}{%
        \alt<#1>{\pgfkeysalso{#2}}{\pgfkeysalso{#3}}
        % \pgfkeysalso doesn't change the path
    },
}

\bibliography{../bib/references}
% \author[J.\ Oaks]{
    %Jamie R.\ Oaks\inst{1}
    Jamie R.\ Oaks
}
% \institute[BIOL 180]{
%     \inst{}%
%         BIOL 180: Introductory Biology
% }



\title[Evolutionary processes \& Speciation]{Evolutionary processes \& Speciation}
% \date{\today}
\date{April 25, 2018}

\begin{document}

\maketitle

\begin{frame}
\frametitle{Today's issues:}
What are the fundamental processes of evolution?

\vspace{2cm}
How do new species form?
\end{frame}

\begin{frame}
\begin{adjustwidth}{-1.5em}{-1.5em}
    \begin{itemize}
        \item<1-> Now we know where variation comes from
        \begin{itemize}
            \item<2-> Ultimately from mutations; sex and meiosis shuffle the variation
        \end{itemize}
        \item<3-> And how it's inherited
        \begin{itemize}
            \item<4-> Alleles are passed to offspring on chromosomes 
        \end{itemize}
    \end{itemize}

    \vspace{2cm}
    \uncover<5->{
    What controls the fate of new alleles introduced into a population via
    mutation?}
\end{adjustwidth}
\end{frame}

\begin{frame}
\begin{adjustwidth}{-1.5em}{-1.5em}
    \textbf{Natural selection:}
    \begin{itemize}
        \item Alleles that confer a reproductive advantage in the current
            environment will tend to increase in frequency over generations 
    \end{itemize}

    \vspace{5mm}
    \uncover<2->{
    \textbf{Genetic drift:}
    \begin{itemize}
        \item Chance variation in reproductive success
        \item Allele frequencies change every generation just due to random
            chance
        \item Drift is random with respect to fitness
    \end{itemize}
    }

    \vspace{5mm}
    \uncover<3->{
    Mutation is always introducing new alleles into a population
    }

    \vspace{5mm}
    \uncover<4->{
    Selection and drift control the fate of those alleles
    }
\end{adjustwidth}
\end{frame}

\begin{frame}
\begin{adjustwidth}{-1.5em}{-1.5em}
    Genetic drift has a stronger affect on allele frequencies in smaller
    populations
    \begin{itemize}
        \item ``Sampling'' fewer gametes each generation, so will have
            larger changes in allele frequencies due to chance
        \item Even beneficial alleles can be lost due to chance
    \end{itemize}

    \vspace{1cm}
    \uncover<2->{
    In larger populations
    \begin{itemize}
        \item Random changes in allele frequencies still happen every
            generation, but changes are smaller
        \item Natural selection is more efficient 
    \end{itemize}
    }
\end{adjustwidth}
\end{frame}

\begin{frame}
\begin{adjustwidth}{-1.5em}{-1.5em}
    When Europeans brought smallpox and measles to North America, up to 90\% of
    Native Americans in some communities died

    \vspace{1cm}
    This caused changes in allele frequencies

    \vspace{1cm}
    Were these changes due to natural selection or genetic drift?
\end{adjustwidth}
\end{frame}


\begin{frame}
\begin{adjustwidth}{-1.5em}{-1.5em}
    Evolution = the change of allele frequencies in a population over time.

    \vspace{5mm}
    The fundamental processes of evolution:
    \begin{itemize}
        \item Mutation
        \item Genetic drift
        \item Natural selection
    \end{itemize}

    \uncover<2->{
    \vspace{5mm}
    Mutation introduces heritable variation
    }

    \uncover<3->{
    \vspace{5mm}
    Drift and selection remove variation
    }

    \uncover<4->{
    \vspace{5mm}
    Drift does so randomly, selection does so nonrandomly
    }
\end{adjustwidth}
\end{frame}

\begin{frame}
    \begin{center}
        {\LARGE\bf How do new species form?}
    \end{center}
\end{frame}

\begin{frame}
\begin{adjustwidth}{-1.5em}{-1.5em}
    What is a species?
    \uncover<2->{
    \begin{itemize}
        \item A group of populations in which evolutionary processes are acting
            independently from other groups of populations
    \end{itemize}
    }

    \vspace{1cm}
    \uncover<3->{
    Speciation occurs when:
    \begin{itemize}
        \item Genetic isolation leads to
        \item Genetic divergence due to mutation, drift, and selection
    \end{itemize}
    }
\end{adjustwidth}
\end{frame}

\begin{frame}
\begin{adjustwidth}{-1.5em}{-1.5em}
    Let's use elephants as an example
    \begin{itemize}
        \item<2-> Imagine an ancestral species of elephant living in Africa
        \item<3-> When Africa moves north to collide with Eurasia, some
            individuals move into Asia, becoming isolated
        \item<4-> Selection and drift now change the allele frequencies in the
            Asian and African populations independently
        \item<5-> Mutations introduce different alleles in Africa and Asia
            independently
        \item<6-> Differences accumulate over time and the populations diverge

    \end{itemize}
\end{adjustwidth}
\end{frame}


% \subsection{Allopatric speciation}

% \subsubsection{Dispersal}

\begin{frame}[t]
    \frametitle{Allopatric (``different land'') Speciation}
    \vspace{-4mm}
    \begin{adjustwidth}{-1.5em}{-1.5em}
        If populations are separated geographically, then evolutionary forces
        will act on them independently.
        \begin{uncoverenv}<2->
        \begin{enumerate}
            \item \textbf{Dispersal}
                \begin{itemize}
                    \item E.g., Iguanas of Anguilla
                        % (\href{http://www.nature.com/news/1998/981015/full/news981015-3.html}{link})
                        $\ldots$ Fiji, Galapagos, $\ldots$
                \end{itemize}
        \end{enumerate}
        \end{uncoverenv}
    \end{adjustwidth}

    \note[item]{1995, Hurricane Luis passed through Lesser Antilles. A floating
        mat of trees harbored 15 iguanas for more than 200 miles from
        Guadeloupe to Anguilla.}
\end{frame}

\begin{frame}[t]
    \begin{adjustwidth}{-1.5em}{-1.5em}
        \begin{columns}
            \column{0.3\linewidth}
            Hawaiian honeycreepers

            \column{0.7\linewidth}
            \begin{figure}
                \begin{center}
                    \includegraphics[width=\columnwidth]{../images/honeycreepers-labeled.jpg}
                    \caption{\tiny \copyright \href{http://www.hdouglaspratt.com/}{H.\ Douglas Pratt}}
                \end{center}
            \end{figure}
        \end{columns}
    \end{adjustwidth}
    \note[item]{Diverse monophyletic group of Hawaiian birds whose closest
        relatives are mainland finches}
\end{frame}

\begin{frame}[t]
    \begin{adjustwidth}{-1.5em}{-1.5em}
    {\small Is this phylogeny of Hawaiian honeycreepers consistent with dispersal?}
        \begin{figure}
            \begin{center}
                \includegraphics[width=\linewidth]{../images/honeycreeper-biogeography.jpg}
                \caption{\shortfullcite{Bromham2003}}
            \end{center}
        \end{figure}
    \end{adjustwidth}
    \note[item]{Yes! Outgroups (orange) are mainland finches. Phylogeny
        consistent with pattern of dispersal down island chain by age of island
        emergence!}
    \note[item]{Hawaiian silverswords and Drosophila show the same pattern}
\end{frame}

% \subsubsection{Vicariance}

\begin{frame}[t]
    \frametitle{Allopatric (``different land'') Speciation}
    \vspace{-4mm}
    \begin{adjustwidth}{-1.5em}{-1.5em}
        If populations are separated geographically, then evolutionary forces
        will act on them independently.
        \begin{enumerate}
            \item \textbf{Dispersal}
            \item \textbf{Vicariance:} Splitting an existing range into fragments
                \begin{itemize}
                    \item E.g., Death Valley pupfish
                \end{itemize}
        \end{enumerate}
    \end{adjustwidth}

    \note[item]{50,000 years ago Death Valley was rainy and had system of
        rivers and lakes. As lakes and rivers shrank, pupfish became isolated
        in tiny spring-fed oases. Each is now a distinct species.  Among rarest
        and most narrowly distributed species!}
\end{frame}


% \subsection{Sympatric speciation}

\begin{frame}[t]
    \frametitle{Sympatric (``same land'') Speciation}
    \vspace{-4mm}
    \begin{adjustwidth}{-1.5em}{-1.5em}
        Can genetic isolation occur if populations are in the same geographic
        area?

        \uncover<2->{
        \vspace{5mm}
        \textbf{Mechanism of isolation:}
        \begin{itemize}
            \item Selection for opposite ``extremes'' of phenotype; hybrid
                individuals with intermediate trait values have low fitness
        \end{itemize}
        }

        \uncover<3->{
        \vspace{5mm}
        \textbf{Mechanism of divergence:}
        \begin{itemize}
            \item Once genetically isolated, evolutionary forces will act
                    independently in each phenotypic group
            \item Differences accumulate via mutation, selection, and drift
        \end{itemize}
        }
    \end{adjustwidth}
\end{frame}

\end{document}

\clickerslide{
\begin{frame}
    \begin{clickerquestion}
        \item 
        \begin{clickeroptions}
            \item 
            \item 
            \item 
            \item 
        \end{clickeroptions}
    \end{clickerquestion}
\end{frame}
}
