\documentclass[letterpaper,11pt]{article}
\pdfpagewidth = 8.5in
\pdfpageheight = 11.0in
\usepackage[left=1in,right=1in,top=1in,bottom=1in]{geometry}
% \usepackage[showframe,left=1in,right=1.01in,top=1in,bottom=1.02in]{geometry}

\pagestyle{plain}
% \usepackage{anysize}
% \papersize{11in}{8.5in}
% \marginsize{1in}{1in}{0.5in}{0.5in}
\pagenumbering{arabic}
\usepackage{setspace}
\usepackage{xspace}
\usepackage[usenames]{color}
\usepackage[fleqn]{amsmath}
\usepackage{graphicx}
\usepackage{framed}
\usepackage{array}
\usepackage{tabulary}
\usepackage{booktabs}
\usepackage{wrapfig}
\usepackage{floatrow}
\floatsetup[table]{font=sf}
\usepackage{ifthen}
\usepackage{multirow}
\usepackage[format=plain, labelsep=period, %justification=raggedright,
            singlelinecheck=true, skip=2pt, font={footnotesize,sf},
            labelfont=bf]{caption}

%% Set up color palettes %%%%%%%%%%%%%%%%%%%%%%%%%%%%%%%%%%%%%%%%%%%%
\definecolor{citescol}{RGB}{194,101,1}
%\definecolor{citescol}{RGB}{73,0,165}
\definecolor{urlscol}{RGB}{0,150,206}
%\definecolor{urlscol}{RGB}{0,107,124}
%\definecolor{linkscol}{RGB}{187,24,0}
\definecolor{linkscol}{RGB}{149,0,207}
%\definecolor{linkscol}{RGB}{73,0,165}
\definecolor{mycol}{RGB}{25,23,191}
\definecolor{outputcol}{RGB}{34,139,34}
\definecolor{tcol}{RGB}{165,0,14}

% Color palette GreenOrange_6 from https://jiffyclub.github.io/palettable/tableau/
\definecolor{pgreen}     {RGB}{50,162,81}
\definecolor{porange}    {RGB}{255,127,15}
\definecolor{pblue}      {RGB}{60,183,204}
\definecolor{pyellow}    {RGB}{255,217,74}
\definecolor{pteal}      {RGB}{57,115,124}
\definecolor{pauburn}    {RGB}{184,90,13}
%%%%%%%%%%%%%%%%%%%%%%%%%%%%%%%%%%%%%%%%%%%%%%%%%%%%%%%%%%%%%%%%%%%%%

%% Set up linking %%%%%%%%%%%%%%%%%%%%%%%%%%%%%%%%%%%%%%%%%%%%%%%%%%%
\usepackage{hyperref}
\hypersetup{pdfborder={0 0 0}, colorlinks=true, urlcolor=pteal, linkcolor=pauburn,
            citecolor=pteal}
%%%%%%%%%%%%%%%%%%%%%%%%%%%%%%%%%%%%%%%%%%%%%%%%%%%%%%%%%%%%%%%%%%%%%

%% Set up bibliobraphy %%%%%%%%%%%%%%%%%%%%%%%%%%%%%%%%%%%%%%%%%%%%%%
% \usepackage{cleveref}
% \usepackage[round]{natbib}
% \bibliographystyle{../bib/sysbio}

\usepackage[backend=bibtex,citestyle=numeric-comp,hyperref=true,firstinits=true,terseinits=true,sorting=none,doi=false,url=true,isbn=false,eprint=false,maxbibnames=20]{biblatex}
% \addbibresource{../bib/references}
% \input{../bib/biblatex-macros}
\DeclareFieldFormat[article,inbook,incollection,inproceedings,thesis]{url}{}

% The following definition is copied from authortitle.bbx/authoryear.bbx
\defbibenvironment{nolabelbib}
  {\list
     {}
     {\setlength{\leftmargin}{\bibhang}%
      \setlength{\itemindent}{-\leftmargin}%
      \setlength{\itemsep}{\bibitemsep}%
      \setlength{\parsep}{\bibparsep}}}
  {\endlist}
  {\item}
%%%%%%%%%%%%%%%%%%%%%%%%%%%%%%%%%%%%%%%%%%%%%%%%%%%%%%%%%%%%%%%%%%%%%

\newcommand{\projecttitle}{Generalizing Bayesian phylogenetics to infer shared evolutionary events\xspace}
\newcommand{\projectcost}{551,169\xspace}
\newcommand{\projectperiod}{1 March 2017 -- 29 February 2020\xspace}
\newcommand{\projectlocation}{Auburn Univeristy, Auburn, Alabama}

\usepackage[compact]{titlesec}
\titleformat{\part}[hang]
    {\Large\scshape\filcenter}
    {\thepart}{.5em}{}[]
\titlespacing*{\part}
    {0ex}{*0}{*0}

\newenvironment{mytitle}
 {\parskip=0pt\par\nopagebreak\centering\LARGE\slshape}
 {\par\noindent\ignorespacesafterend}
\newenvironment{tightCenter}
 {\parskip=0pt\par\nopagebreak\centering}
 {\par\noindent\ignorespacesafterend}

\titleformat{\section}[hang]
    {\normalfont\Large\bfseries}
    {\thesection}{1em}{}[]
\titlespacing*{\section}
    {0ex}{1.0ex plus .1ex minus .1ex}{0ex}
    % {0ex}{0ex}{0ex}

\titleformat{\subsection}[hang]
    {\normalfont\large\bfseries}
    {\thesubsection}{1em}{}[]
\titlespacing*{\subsection}
    {0ex}{1.0ex plus .1ex minus .1ex}{0ex}
    % {0ex}{0ex}{0ex}

\titleformat{\subsubsection}[hang]
    {\normalfont\normalsize\bfseries}
    {\thesubsubsection}{1em}{}[]
\titlespacing*{\subsubsection}
    {0ex}{1.0ex plus .1ex minus .1ex}{0ex}

\titleformat{\paragraph}[runin]
    {\normalsize\slshape\bfseries}
    {}{}{}[---]
\titlespacing{\paragraph}
    {0ex}{0.5ex plus .1ex minus .1ex}{0.5ex}

\titleformat{\subparagraph}[runin]
    {\normalsize\slshape\bfseries}
    {}{}{}[---]
\titlespacing{\subparagraph}
    {\parindent}{0ex}{0.5ex}

% Define tightly formated list environments via enumitem
\usepackage[inline]{enumitem}
\newenvironment{tightItemize}{%
\begin{itemize}[noitemsep, topsep=0pt, parsep=0pt, partopsep=0pt]}
{\end{itemize}}

\newenvironment{veryTightItemize}{%
\begin{itemize}[noitemsep, topsep=0pt, parsep=0pt, partopsep=0pt, leftmargin=*]}
{\end{itemize}}

\newenvironment{tightEnumerate}{%
\begin{enumerate}[noitemsep, topsep=0pt, parsep=0pt, partopsep=0pt]}
{\end{enumerate}}

\newenvironment{veryTightEnumerate}{%
\begin{enumerate}[noitemsep, topsep=0pt, parsep=0pt, partopsep=0pt, leftmargin=*]}
{\end{enumerate}}

\newenvironment{inlineEnumerate}{%
\begin{enumerate*}[label=(\arabic*), itemjoin={{, }}, itemjoin*={{, and }}]} %after={.}]}
{\end{enumerate*}}

% \newenvironment{cellItemize}{%
% \begin{itemize*}[itemjoin={\newline}]}
% {\end{itemize*}}
\newenvironment{cellItemize}{%
\begin{itemize}[noitemsep, topsep=0pt, parsep=0pt, partopsep=0pt, leftmargin=*]}
{\end{itemize}}

% Description environment
%\newcommand\myDescriptionLabel[1]{\hspace{\labelsep}#1:}
\newcommand\myDescriptionLabel[1]{#1}
\newenvironment{myDescription}[1][1in]{
  \let\descriptionlabel\myDescriptionLabel
  \begin{description}[labelsep=1em, align=left, labelwidth=#1, labelindent=0cm]
    \setlength{\itemsep}{0.15em}
    \setlength{\parskip}{0pt}
    \setlength{\parsep}{0.5em}}
  {\end{description}}

%% Create flexible CV item environment
\newlength{\defaultTabColSep}
\newlength{\CVItemFirstCol}
\newlength{\CVItemFirstColDefault}
\setlength\CVItemFirstColDefault{2cm}
\newlength{\CVItemSecondCol}
\newcommand{\cvitem}[3]{
    \ifthenelse{\equal{#1}{}}{
        \setlength\CVItemFirstCol{\CVItemFirstColDefault}}{
        \setlength\CVItemFirstCol{#1}}
    \setlength\CVItemSecondCol{\textwidth}
    \addtolength{\CVItemSecondCol}{-\CVItemFirstCol}
    \addtolength{\CVItemSecondCol}{-2\tabcolsep}
    % \begin{tabular}{ @{}p{\CVItemFirstCol} >{\begin{minipage}[t]{\CVItemSecondCol}\raggedright}l<{\end{minipage}}}
    \begin{tabular}{ @{}p{\CVItemFirstCol} >{\begin{minipage}[t]{\CVItemSecondCol}}l<{\end{minipage}}}
         #2 & #3
    \end{tabular}}

\newcommand{\descriptionitem}[3]{
    \setlength\defaultTabColSep{\tabcolsep}
    \setlength\tabcolsep{0pt}
    \ifthenelse{\equal{#1}{}}{
        \setlength\CVItemFirstCol{\CVItemFirstColDefault}}{
        \setlength\CVItemFirstCol{#1}}
    \setlength\CVItemSecondCol{\textwidth}
    \addtolength{\CVItemSecondCol}{-\CVItemFirstCol}
    \addtolength{\CVItemSecondCol}{-3\tabcolsep}
    % \begin{tabular}{ @{}p{\CVItemFirstCol} >{\begin{minipage}[t]{\CVItemSecondCol}\raggedright}l<{\end{minipage}}}
    \begin{tabular}{ @{}p{\CVItemFirstCol} >{\begin{minipage}[t]{\CVItemSecondCol}}l<{\end{minipage}}}
         #2: & #3
    \end{tabular}
    \setlength\tabcolsep{\defaultTabColSep}}

\newcommand{\timeitem}[3]{
    \setlength\defaultTabColSep{\tabcolsep}
    \setlength\tabcolsep{0pt}
    \ifthenelse{\equal{#1}{}}{
        \setlength\CVItemFirstCol{\CVItemFirstColDefault}}{
        \setlength\CVItemFirstCol{#1}}
    \setlength\CVItemSecondCol{\textwidth}
    \addtolength{\CVItemSecondCol}{-\CVItemFirstCol}
    \addtolength{\CVItemSecondCol}{-3\tabcolsep}
    \renewcommand{\arraystretch}{1.06}
    % \begin{tabular}{ @{}p{\CVItemFirstCol} >{\begin{minipage}[t]{\CVItemSecondCol}\raggedright}l<{\end{minipage}}}
    \begin{tabular}{ @{}p{\CVItemFirstCol} >{\begin{minipage}[t]{\CVItemSecondCol}}l<{\end{minipage}}}
        \textbf{\textsl{#2}} & #3
    \end{tabular}
    \setlength\tabcolsep{\defaultTabColSep}}

%% macro to make long strings breakable over lines
\makeatletter
\def\breakable#1{\xHyphen@te#1$\unskip}
\def\xHyphen@te{\@ifnextchar${\@gobble}{\sw@p{\allowbreak{}\xHyphen@te}}}
% \def\xHyphen@te{\@ifnextchar${\@gobble}{\sw@p{\hskip 0pt plus 1pt\xHyphen@te}}}
\def\sw@p#1#2{#2#1}
\makeatother
